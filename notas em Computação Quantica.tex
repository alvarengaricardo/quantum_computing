\documentclass[12pt]{article}
\usepackage[utf8]{inputenc} % Pacote para acentuação gráfica
%\usepackage[T1]{fontenc}
\usepackage[brazil]{babel} % nomes das estruturas em pt-br
%\usepackage{hyperref}
\usepackage{indentfirst} % indenta primeiro paragráfo após título
\usepackage{setspace} % pacote para alterar espaçamento entre linhas
%\setlength{\parindent}{1cm} % define o tamanho da indentação
%\setlength{\parskip}{0.3cm} % define o espaçamento vertical entre parágrafos
\usepackage[top = 2cm, left = 2cm, bottom = 2cm, right = 2cm]{geometry} % define as margens do documento
\usepackage{fancyhdr} % pacote para numeração de páginas

\usepackage{xcolor}
% Definindo novas cores
\definecolor{verde}{rgb}{0.25,0.5,0.35}
\definecolor{jpurple}{rgb}{0.5,0,0.35}
% Configurando layout para mostrar codigos Java
\usepackage{listings}
\lstset{
	language=Java,
	basicstyle=\ttfamily\small,
	keywordstyle=\color{jpurple}\bfseries,
	stringstyle=\color{red},
	commentstyle=\color{verde},
	morecomment=[s][\color{blue}]{/**}{*/},
	extendedchars=true,
	showspaces=false,
	showstringspaces=false,
	numbers=left,
	numberstyle=\tiny,
	breaklines=true,
	backgroundcolor=\color{cyan!10},
	breakautoindent=true,
	captionpos=b,
	xleftmargin=0pt,
	tabsize=4
}
\pagestyle{empty}

\begin{document}
	
\title{\textbf{{\Huge Notas em Computação Quântica}}} % Título
\author{\textbf{{\Large Ricardo Alvarenga}}} % Autor
\date{\textbf{{\Large 2024}}} % Data
\maketitle % Criar
\thispagestyle{empty} % remove número da página
\newpage

\setcounter{page}{1} % reset contador de página
\pagenumbering{Roman}
\tableofcontents % cria sumário
\pagestyle{fancy}
\fancyfoot[C]{\thepage} % Adiciona o número da página no centro do rodapé
\newpage

\setcounter{page}{1} % reset contador de página
\pagenumbering{arabic}
\pagestyle{fancy}
\fancyfoot[C]{\thepage}

\section{Álgebra Linear}
\subsection{Espaço Vetorial}



	
\end{document}